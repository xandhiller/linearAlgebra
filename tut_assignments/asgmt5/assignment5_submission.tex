\documentclass{article}
\author{Alex Hiller, 11850637}
\title{Linear Algebra, Assignment 5}
% Type-setting
\setlength{\parindent}{0cm}
\setlength{\parskip}{0.125cm}
\pagenumbering{gobble}
\usepackage[margin=2.5cm]{geometry} % Formatting
\usepackage{amsmath}      % Mathematics
\usepackage{amssymb}      % Mathematics
\usepackage{listings}     % Listings
\usepackage{esint}        % Mathematics
\usepackage{color}        % Listings
\usepackage{courier}      % Listings
\usepackage{circuitikz}   % Circuits
\usepackage{titlesec}     % Section Formatting
\usepackage{stmaryrd}               % \mapsfrom arrow. 
\usepackage{svg}
\setsvg{inkscape=inkscape -z -D}
\input{/home/polluticorn/GitHub/texTemplates/texMacros}
% Section formatting
\titleformat{\section}{\huge \bfseries}{}{0em}{}[]
\titleformat{\subsection}{\Large \bfseries}{}{0em}{}
\titleformat{\subsubsection}{\bfseries}{}{0em}{}
%%%%%%%%%%%%%%%%%%%%%%%%%%%%%%%%%%%%%%%%%%%%%%%%%%%%%%%%%%%%%%%%%%%%%%%%%%%%%%%%

\begin{document}
\maketitle
\section{Question 1} 
\[%
    \mathbf{A}
    =
    \begin{bmatrix}
        1 & -2 & 7 & -6 \\
		-2 & -1 & -9 & 7 \\
		1 & 13 & -8 & 9 \\
    \end{bmatrix}
\]%
Basis for a vector space spanned by the columns of $ \mathbf{A} $
First, let's find the vector space spanned by $ \mathbf{A}  $:
\[%
    \text{rref} (\mathbf{A}) 
    =
    \begin{bmatrix}
        1 & 0 & 5 & -4 \\
		0 & 1 & -1 & 1 \\
		0 & 0 & 0 & 0 \\
    \end{bmatrix}
    \quad \Rightarrow \quad
    \text{rank} ( \mathbf{A} ) = 2
    \quad \Rightarrow \quad
    \text{Span} \{ \mathbf{A} \} = \R^{2}
\]%
A valid basis for this vector space is that given in the cartesian system:
\[%
    \mathbf{a}_1 
    =
    \begin{bmatrix}
        1 \\0 \\
    \end{bmatrix}
    \quad
    \mathbf{a}_2
    =
    \begin{bmatrix}
        0 \\ 1 \\
    \end{bmatrix}
    \quad
    \Rightarrow 
    \quad
    \text{Span} \{ \mathbf{a}_{1} , \mathbf{a}_{2}  \} = \R^{2}
\]%

\section{Question 2} 
With:
\[%
    B = 
    \left\{ 
    \begin{bmatrix}
        1 \\ 2 \\ 3 \\ 4 \\
    \end{bmatrix},
    \begin{bmatrix}
        1 \\ 2 \\ 3 \\ 0 \\
    \end{bmatrix},
    \begin{bmatrix}
        1 \\ 2 \\ 0 \\ 0 \\
    \end{bmatrix},
    \begin{bmatrix}
        1 \\ 0 \\ 0 \\ 0 \\
    \end{bmatrix}
    \right\}
\]%
We can prove that $B$ is a basis in $ \R^{4} $ if it is linearly independent and
if it spans $ \R^{4} $.

\subsection{(a) Linear Independence:} 
Forming an augmented matrix:
\[%
    \begin{bmatrix}
       1 & 1 & 1 & 1 \\
       2 & 2 & 2 & 0 \\
       3 & 3 & 0 & 0 \\
       4 & 0 & 0 & 0 \\
    \end{bmatrix}
\]%
Through row reduction we can reduce this to:
\[%
    \begin{bmatrix}
        1 & 0 & 0 & 0\\
        0 & 1 & 0 & 0\\
        0 & 0 & 1 & 0\\
        0 & 0 & 0 & 1\\		
    \end{bmatrix}
\]%
The matrix has 4 pivots, rank = 4 and hence must span $ \R^{4} $.

\subsection{(b) Find $ \mathbf{x} $} 
If you have a basis s.t.
\[%
    \mathbf{B} = \left\{ \mathbf{b_1}, \ \ldots \ , \mathbf{b_{n}}  \right\}
\]%
We can redefine $ \mathbf{B} $ to be:
\[%
    \mathbf{B} = 
    \begin{bmatrix}
        \mathbf{b}_{1} \ \ldots \ \mathbf{b}_{n}
    \end{bmatrix}
\]%
Then $ \mathbf{x} $ is expressed as $ \left[ \mathbf{x} \right]_{B}  $ via the
equation:
\[%
    \mathbf{x} =  \mathbf{B}\left[ \mathbf{x} \right]_{B}
\]%
As a corrollary, it also then follows that:
\[%
    \left[ \mathbf{x} \right]_{B} = \mathbf{B}^{-1}\mathbf{x}
\]%

Solving $ \mathbf{x} = \mathbf{B} \left[ \mathbf{x} \right]_{B}  $  :
\[%
    \mathbf{x} = 
    \begin{bmatrix}
       1 & 1 & 1 & 1 \\
       2 & 2 & 2 & 0 \\
       3 & 3 & 0 & 0 \\
       4 & 0 & 0 & 0 \\
    \end{bmatrix}
    \begin{bmatrix}
        -1 \\		
        0 \\
		0 \\
		2 \\
    \end{bmatrix}
    =
    \begin{bmatrix}
        1 \\		
        -2 \\
		-3 \\
		-4 \\
    \end{bmatrix}
\]%

\subsection{(c) Find $ \left[ \mathbf{y} \right]_{B} $} 
\[%
    \left[ \mathbf{y} \right]_{B} = \mathbf{B}^{-1}\mathbf{y}
\]%
Inverting $ \mathbf{B} $, we get:
\[%
    \mathbf{B}^{-1} = 
    \begin{bmatrix}
        0 & 0 & 0 & \frac{1}{4}  \vspace{2mm}\\ 
        0 & 0 & \frac{1}{3}  & -\frac{1}{4} \vspace{2mm} \\ 
		0 & \frac{1}{2}  & -\frac{1}{3}  & 0 \vspace{2mm} \\ 
		1 & -\frac{1}{2}  & 0 & 0 \\		
    \end{bmatrix}
\]%
Evaluating:
\[%
    \left[ \mathbf{y} \right]_{B} = \mathbf{B}^{-1}\mathbf{y} 
    =
    \begin{bmatrix}
        0 & 0 & 0 & \frac{1}{4}  \vspace{2mm}\\ 
        0 & 0 & \frac{1}{3}  & -\frac{1}{4} \vspace{2mm} \\ 
		0 & \frac{1}{2}  & -\frac{1}{3}  & 0 \vspace{2mm} \\ 
		1 & -\frac{1}{2}  & 0 & 0 \\		
    \end{bmatrix}
    \begin{bmatrix}
        10 \\		
        12 \\
		9 \\
		4 \\
    \end{bmatrix} 
    =
    \begin{bmatrix}
        1 \\		
        2 \\
		3 \\
		4 \\
    \end{bmatrix} 
\]%


\section{Question 3}
If the linear transformation $ \mathbf{R} $ is:
\[%
    \begin{bmatrix}
        2 & 0 \\
		0 & -3 \\		
    \end{bmatrix}
\]%
Let's define the unit square as the set:
\[%
    S = \left\{ s_1, s_2, s_3, s_4 \right\}
\]%
Where the elements of the set are:
\[%
    s_1 = 
    \begin{bmatrix}
        0 \\ 0 \\
    \end{bmatrix}
    \quad
    s_2 =
    \begin{bmatrix}
        1 \\ 0 \\
    \end{bmatrix}
    \quad
    s_3 = 
    \begin{bmatrix}
        1 \\ 1 \\
    \end{bmatrix}
    \quad
    s_4 = 
    \begin{bmatrix}
        0 \\ 1 \\
    \end{bmatrix}
\]%
Then mapping each element of $ S $ with $ \mathbf{R} $, we get:
\[%
    \mathbf{R} \ s_1 = 
    \begin{bmatrix}
        2 & 0 \\
		0 & -3 \\		
    \end{bmatrix}
    \begin{bmatrix}
        0 \\ 0 \\
    \end{bmatrix}
    =
    \begin{bmatrix}
        0 \\ 0 \\
    \end{bmatrix}
\]%
\[%
    \mathbf{R} \ s_2 = 
    \begin{bmatrix}
        2 & 0 \\
		0 & -3 \\		
    \end{bmatrix}
    \begin{bmatrix}
        1 \\ 0 \\
    \end{bmatrix}
    = 
    \begin{bmatrix}
        2 \\ 0 \\
    \end{bmatrix}
\]%
\[%
    \mathbf{R} \ s_3 = 
    \begin{bmatrix}
        2 & 0 \\
		0 & -3 \\		
    \end{bmatrix}
    \begin{bmatrix}
        1 \\ 1 \\
    \end{bmatrix}
    =
    \begin{bmatrix}
        2 \\ -3 \\ 
    \end{bmatrix}
\]%
\[%
    \mathbf{R} \ s_3 = 
    \begin{bmatrix}
        2 & 0 \\
		0 & -3 \\		
    \end{bmatrix}
    \begin{bmatrix}
        0 \\ 1 \\
    \end{bmatrix}
    =
    \begin{bmatrix}
        0 \\ -3 \\
    \end{bmatrix}
\]%
Illustrated, it looks like so:
\begin{figure}[!htbp]
    \centering
    \includesvg[width=0.1\textwidth]{transformation} % No file extension.
    \caption{Blue is the unit square, red is the unit square transformed by  $
    \mathbf{R} $}
\end{figure}


\section{Question 4} 
\subsection{(a) Domain and Codomain of $ \mathcal{T} $} 
Because $ \mathbf{A} $ is $ 4 \times 3 $ with respect to its size:
\[%
    \mathbf{A} : \R^{4} \mapsto \R^{3}
\]%

Therefore 
\[%
    \text{Domain} = \R^{4}
\]%
\[%
    \text{Codomain} = \R^{3}
\]%

\subsection{(b) Range of $ \mathcal{T} $} 

The issue is that the columns of $ \mathbf{A} $ are linearly dependent. This
means that the columns do not span $ \R^{3} $, it is instead a plane in 3D
space.

\subsection{(c) Images of Vectors} 
\[%
    \mathcal{T}(\mathbf{v})=
    \begin{bmatrix}
        1 & -2 & 7 & -6 \\
		-2 & -1 & -9 & 7 \\
		1 & 13 & -8 & 9 \\
    \end{bmatrix}
    \begin{bmatrix}
        1 \\
		1 \\
		1 \\
		1 \\		
    \end{bmatrix}
    = 
    \begin{bmatrix}
        0 \\
		-5 \\
		15 \\		
    \end{bmatrix}
\]%
\[%
    \mathcal{T}(\mathbf{u})= 
    \begin{bmatrix}
        1 & -2 & 7 & -6 \\
		-2 & -1 & -9 & 7 \\
		1 & 13 & -8 & 9 \\
    \end{bmatrix}
    \begin{bmatrix}
        0 \\
		1 \\
		2 \\
		2 \\		
    \end{bmatrix}
    =
    \begin{bmatrix}
        0 \\
		-5 \\
		15 \\		
    \end{bmatrix}
\]%
\[%
    \mathcal{T}(\mathbf{w})=
    \begin{bmatrix}
        1 & -2 & 7 & -6 \\
		-2 & -1 & -9 & 7 \\
		1 & 13 & -8 & 9 \\
    \end{bmatrix}
    \begin{bmatrix}
        2 \\
		1 \\
		0 \\
		0 \\		
    \end{bmatrix}
    =
    \begin{bmatrix}
        0 \\
		-5 \\
		15 \\		
    \end{bmatrix}
\]%

All the answers lie in a plane because the range of $ \mathbf{A} $ is $ \R^{2}
$. 

However, couldn't tell you why they are the same point in the plane\ \ldots \



\end{document}

