\documentclass{article}
\author{Alex Hiller (11850637)}
\title{Linear Algebra, Assignment 6}
% Type-setting
\setlength{\parindent}{0cm}
\setlength{\parskip}{0.125cm}
\pagenumbering{gobble}
\usepackage[margin=2.5cm]{geometry} % Formatting
\usepackage{amsmath}      % Mathematics
\usepackage{amssymb}      % Mathematics
\usepackage{listings}     % Listings
\usepackage{esint}        % Mathematics
\usepackage{color}        % Listings
\usepackage{courier}      % Listings
\usepackage{circuitikz}   % Circuits
\usepackage{titlesec}     % Section Formatting
\usepackage{stmaryrd}               % \mapsfrom arrow. 
\usepackage{svg}
\usepackage{mathtools}
\setsvg{inkscape=inkscape -z -D}
\input{/home/polluticorn/GitHub/texTemplates/texMacros}
% Section formatting
\titleformat{\section}{\huge \bfseries}{}{0em}{}[]
\titleformat{\subsection}{\Large \bfseries}{}{0em}{}
\titleformat{\subsubsection}{\bfseries}{}{0em}{}

%%%%%%%%%%%%%%%%%%%%%%%%%%%%%%%%%%%%%%%%%%%%%%%%%%%%%%%%%%%%%%%%%%%%%%%%%%%%%%%%

\begin{document}
\maketitle

\clearpage
\section{Question 1} 
\subsection{Part (a)} 
If we define $ \mathbf{x} $ as the unit square, $ \mathbf{y} $ as the
image of $ \mathbf{x} $ and coordinates to be given as: $ \begin{bmatrix}
    x-\text{coordinate} \\y-\text{coordinate}\\
\end{bmatrix} $ then the mapping is doing the following:

\[%
    \mathbf{x} \coloneqq \left\{ 
    \begin{bmatrix}
        0 \\ 0 \\
    \end{bmatrix}
    ,
    \begin{bmatrix}
        0\\1\\
    \end{bmatrix}
    ,
    \begin{bmatrix}
        1\\1\\
    \end{bmatrix}  
    ,
    \begin{bmatrix}
        1\\0\\
    \end{bmatrix}
    \right\}
    
\]%

\[%
    \mathcal{T} (\mathbf{x})= \mathbf{y}
\]%

\[%
    \mathbf{y} \coloneqq  \left\{ 
    \begin{bmatrix}
        0 \\ 0 \\
    \end{bmatrix}
    ,
    \begin{bmatrix}
        -1\\2\\
    \end{bmatrix}
    ,
    \begin{bmatrix}
        -3\\3\\
    \end{bmatrix}  
    ,
    \begin{bmatrix}
        -2\\1\\
    \end{bmatrix}
    \right\}
\]%

We know that a transformation of $ \mathcal{T}  $ can be expressed as a matrix
by:
\[%
    \mathbf{T}=
    \begin{bmatrix}
        \mathcal{T}(\mathbf{e_{1}})  \ | \ \mathcal{T} (\mathbf{e_{2}})
    \end{bmatrix}
\]%

Fortunately, we have these given to us in our question:
\[%
    \mathbf{T} =
    \begin{bmatrix}
        \mathcal{T}\left(\begin{bmatrix} 1 \\ 0 \\ \end{bmatrix}\right) \ \bigg| \ 
        \mathcal{T}\left(\begin{bmatrix} 0\\1\\ \end{bmatrix}\right)
    \end{bmatrix}
    =
    \begin{bmatrix}
        -2 & -1 \\
		1 &  2  \\		
    \end{bmatrix}
\]%

\subsection{Part (b)} 
Verifying the image of the top right corner:
\[%
    \mathbf{T}
    \begin{bmatrix}
        1\\1\\
    \end{bmatrix}
    =
    \begin{bmatrix}
        -2 & -1 \\
		1 &  2  \\		
    \end{bmatrix}
    \begin{bmatrix}
        1\\1\\
    \end{bmatrix}
    =
    \begin{bmatrix}
        -3\\3\\
    \end{bmatrix}   
\]%
Which is as expected.

Please turn over for parts (c) and (d).

\clearpage
\subsection{Part (c) and (d)} 
Now we are to apply the transformation to $ \mathbf{T}\mathbf{x} $, (note that
$ \mathbf{T}\mathbf{x} $ is the image of $ \mathbf{x} $ ) leading to:
\[%
    \mathbf{T}\mathbf{T}\mathbf{x}
    =
    \begin{bmatrix}
        -2 & -1 \\
		1 &  2  \\		
    \end{bmatrix}
    \begin{bmatrix}
        -2 & -1 \\
		1 &  2  \\		
    \end{bmatrix}
    \begin{bmatrix}
        1\\1\\
    \end{bmatrix}
\]%

\[%
    \mathbf{T}^{2} 
    =
    \begin{bmatrix}
        3 & 0 \\
        0 & 3 \\
    \end{bmatrix}
\]%

This is a dilation, a stretching of the original unit square to three times its
dimension, depicted below.

\begin{figure}[!htbp]
    \centering
    \includesvg[width=0.2\textwidth]{dilation} % Requires no file extension.
    \caption{A dilation of the unit square by a factor of 3, using $
    \mathbf{T}^{2}. $}
\end{figure}

\clearpage
\section{Question 2} 
\subsection{Part (a)} 
If a $ 2 \times 2 $ matrix has two pivots, then it spans $ \R^{2} $.

Putting
\[%
    \mathbf{P}_{\mathcal{B}} = 
    \begin{bmatrix}
        \mathbf{b}_{1} \ \mathbf{b}_{2} 
    \end{bmatrix}
    =
    \begin{bmatrix}
        -2 & 7 \\
        1 & -2 \\
    \end{bmatrix}
\]%
Then:
\[%
    \text{rref} (\mathbf{\mathbf{P}_{\mathcal{B}}}) 
    =
    \begin{bmatrix}
        1 & 0 \\
        0 & 1 \\
    \end{bmatrix}
\]%
Showing us that $ \mathbf{P}_{\mathcal{B}} $ spans $ \R^{2} $.

Similarly:
\[%
    \mathbf{P}_{\mathcal{C}} = 
    \begin{bmatrix}
        \mathbf{c}_{1} \ \mathbf{c}_{2} 
    \end{bmatrix}
    =
    \begin{bmatrix}
        -1 & 4 \\
        -1 & 1 \\
    \end{bmatrix}
\]%
Again:
\[%
    \text{rref} (\mathbf{\mathbf{P}_{\mathcal{C}}}) 
    =
    \begin{bmatrix}
        1 & 0 \\
        0 & 1 \\
    \end{bmatrix}
\]%
$ \therefore $ The two bases span $ \R^{2} $.

\subsection{Part (b)} 
\[%
    \mathbf{x}= \mathbf{P}_{\mathcal{B}}[\mathbf{x}]_{\mathcal{B}}
    \quad
    \Rightarrow 
    \quad
    [\mathbf{x}]_{\mathcal{B}} = \mathbf{P}_{\mathcal{B}}^{-1} \ \mathbf{x}
\]%
\[%
    \mathbf{P}_{\mathcal{B}} 
    = 
    \begin{bmatrix}
        -2 & 7 \\
        1 & -2 \\
    \end{bmatrix}
    \quad
    \Rightarrow 
    \quad
    \mathbf{P}_{\mathcal{B}}^{-1}
    = 
    \begin{bmatrix}
        \ \frac{2}{3}  & \frac{7}{3} \ \vspace{2mm} \\ \vspace{2mm}
		\ \frac{1}{3}  & \frac{2}{3} \  \\		
    \end{bmatrix}
\]%

Consequently:
\[%
    \begin{bmatrix}
        \ \frac{2}{3}  & \frac{7}{3} \ \vspace{2mm} \\ \vspace{2mm}
		\ \frac{1}{3}  & \frac{2}{3} \  \\		
    \end{bmatrix}
    \begin{bmatrix} 1 \\ 1 \\ \end{bmatrix}
    = 
    [\mathbf{x}]_{\mathcal{B}}
    =
    \begin{bmatrix} 3 \\ 1 \\ \end{bmatrix}
\]%

\subsection{Part (c)} 


We know that
\[%
    \mathbf{x}= \mathbf{P_{\mathcal{B}}}[\mathbf{x}]_{\mathcal{B}} 
\]%
And that
\[%
    \mathbf{x}= \mathbf{P_{\mathcal{C}}}[\mathbf{x}]_{\mathcal{C}} 
\]%
Hence
\[%
    \mathbf{P_{\mathcal{B}}}[\mathbf{x}]_{\mathcal{B}} 
    =
    \mathbf{P_{\mathcal{C}}}[\mathbf{x}]_{\mathcal{C}} 
\]%

\[%
    [\mathbf{x}]_{\mathcal{B}} 
    =
    \mathbf{P_{\mathcal{B}}}^{-1}
    \mathbf{P_{\mathcal{C}}} 
    [\mathbf{x}]_{\mathcal{C}} 
\]%

\[%
    \left(
    \mathbf{P_{\mathcal{B}}}^{-1}
    \mathbf{P_{\mathcal{C}}} 
    \right)
    [\mathbf{x}]_{\mathcal{C}} 
    = 
    \mathbf{P}_{\mathcal{B}\mapsfrom \mathcal{C}}
    [\mathbf{x}]_{\mathcal{C}} 
\]%

So we have in conclusion:

\[%
    [\mathbf{x}]_{\mathcal{B}}
    =
    \mathbf{P}_{\mathcal{B}\mapsfrom \mathcal{C}} [\mathbf{x}]_{\mathcal{C}} 
\]%

Where $ \mathbf{P}_{\mathcal{B}\mapsfrom \mathcal{C}} $ is a matrix that maps
from basis $ \mathcal{C} $ to basis $ \mathcal{B} $.

We of course want the exact inverse of that mapping:

\[%
    (\mathbf{P}_{\mathcal{B}\mapsfrom \mathcal{C}})^{-1}
    = 
    \mathbf{P}_{\mathcal{C}\mapsfrom \mathcal{B}}
    = 
    \left( \mathbf{P_{\mathcal{C}}}^{-1} \mathbf{P_{\mathcal{B}}} \right)
\]%

Another thing we know is that:

\[%
    \mathbf{P}_{\mathcal{B}} 
    = 
    \begin{bmatrix} \mathbf{b}_{1} \ \mathbf{b}_{2} \\ \end{bmatrix}
    =
    \begin{bmatrix} -2 & 7 \\ 1 & -2 \\ \end{bmatrix}
\]%

\[%
    \mathbf{P}_{\mathcal{C}} 
    = 
    \begin{bmatrix} \mathbf{c}_{1} \ \mathbf{c}_{2} \\ \end{bmatrix}
    =
    \begin{bmatrix} -1 & 4 \\ -1 & 1 \\ \end{bmatrix}
\]%

Hence:

\[%
    [\mathbf{x}]_{\mathcal{C}} = \mathbf{P}_{\mathcal{C}}^{-1}
    \mathbf{P}_{\mathcal{B}} [\mathbf{x}]_{\mathcal{B}}
\]%

\[%
    [\mathbf{x}_{\mathcal{C}}] = 
    \begin{bmatrix} -1 & 4 \\ -1 & 1 \\\end{bmatrix}^{-1}
    \begin{bmatrix} -2 & 7 \\ 1 & -2 \\ \end{bmatrix}
    \begin{bmatrix} 3 \\ 1\\ \end{bmatrix}
    =
    \begin{bmatrix} -1 \\ 0 \\ \end{bmatrix}
\]%


\subsection{Part (d)} 
Again:

\[%
    \mathbf{y} = \mathbf{P}_{\mathcal{C}}[\mathbf{y}]_{\mathcal{C}}
\]%

\[%
    [\mathbf{y}]_{\mathcal{C}} = \mathbf{P}_{\mathcal{C}}^{-1}\mathbf{y}
\]%

\[%
    [\mathbf{y}]_{\mathcal{C}} =
    \begin{bmatrix}-1&4\\-1&1\end{bmatrix}^{-1}
    \begin{bmatrix}5\\-1\end{bmatrix}
    =
    \begin{bmatrix}3\\2\end{bmatrix}
\]%

\subsection{Part (e)} 

\[%
    \mathbf{P}_{\mathcal{B}\mapsfrom \mathcal{C}}
    =
    \mathbf{P}_{\mathcal{B}}^{-1} \mathbf{P}_{\mathcal{C}}
\]%

\[%
    [\mathbf{y}]_{\mathcal{B}} 
    =
    \mathbf{P}_{\mathcal{B}\mapsfrom \mathcal{C}}
    [\mathbf{y}]_{\mathcal{C}}
    =
    \begin{bmatrix} -2 & 7 \\ 1 & -2 \\ \end{bmatrix}^{-1}
    \begin{bmatrix} -1 & 4 \\ -1 & 1 \\\end{bmatrix}
    \begin{bmatrix} 3 \\ 2\\ \end{bmatrix}
    =
    \begin{bmatrix} 1\\1 \end{bmatrix}
\]%

\clearpage
\section{Question 3} 
\subsection{Part (a)} 
\[%
    \mathbf{b}_{1} 
    =
    -\mathbf{c}_{1} + 3 \mathbf{c}_{2} 
    = 
    \begin{bmatrix} \mathbf{c}_{1} \ \mathbf{c}_{2}\end{bmatrix}
    \begin{bmatrix} -1 \\ 3  \end{bmatrix}
\]%

\[%
    \mathbf{b}_{2} = 2 \mathbf{c}_{1} - 4 \mathbf{c}_{2}
    =
    \begin{bmatrix} \mathbf{c}_{1} \ \mathbf{c}_{2}\end{bmatrix}
    \begin{bmatrix} 2 \\ -4 \end{bmatrix}
\]%

\[%
    \mathbf{P}_{\mathcal{B}} 
    =
    \begin{bmatrix} \mathbf{b}_{1} \ \mathbf{b}_{2} \end{bmatrix} 
    =
    \begin{bmatrix} \
    \begin{bmatrix} \mathbf{c}_{1} \ \mathbf{c}_{2}\end{bmatrix}
    \begin{bmatrix} -1 \\ 3  \end{bmatrix}
    \ \bigg| \
    \begin{bmatrix} \mathbf{c}_{1} \ \mathbf{c}_{2}\end{bmatrix}
    \begin{bmatrix} 2 \\ -4 \end{bmatrix}
    \
    \end{bmatrix}
\]%

\[%
    \mathbf{P}_{\mathcal{B}}
    = 
    \begin{bmatrix} \mathbf{c}_{1} \ \mathbf{c}_{2}\end{bmatrix}
    \begin{bmatrix} \ -1 & 2\\ 3 & -4 \ \end{bmatrix}
    = 
    \mathbf{P}_{\mathcal{C}}
    \begin{bmatrix} \ -1 & 2\\ 3 & -4 \ \end{bmatrix}
\]%

\[%
    \mathbf{P}_{\mathcal{B}}
    = 
    \mathbf{P}_{\mathcal{C}}
    \begin{bmatrix} \ -1 & 2\\ 3 & -4 \ \end{bmatrix}
\]%

\[%
    \mathbf{P}_{\mathcal{B}}
    \ \mathbf{I}
    = 
    \mathbf{P}_{\mathcal{C}}
    \begin{bmatrix} \ -1 & 2\\ 3 & -4 \ \end{bmatrix}
    \ \mathbf{I}
\]%

Setting $ [\mathbf{x}]_{\mathcal{B}} \coloneqq \mathbf{I} $ we get:

\[%
    \mathbf{P}_{\mathcal{B}}
    [\mathbf{x}]_{\mathcal{B}}
    = 
    \mathbf{P}_{\mathcal{C}}
    \begin{bmatrix} \ -1 & 2\\ 3 & -4 \ \end{bmatrix}
    [\mathbf{x}]_{\mathcal{B}}
\]%

So this must mean that our mystery matrix is the change of basis matrix: $
\mathbf{P}_{\mathcal{C}\mapsfrom \mathcal{B}} $

\[%
    \mathbf{P}_{\mathcal{B}}
    [\mathbf{x}]_{\mathcal{B}}
    = 
    \mathbf{P}_{\mathcal{C}}
    (\mathbf{P}_{\mathcal{C}\mapsfrom \mathcal{B}}) 
    [\mathbf{x}]_{\mathcal{B}}
    = 
    \mathbf{P}_{\mathcal{C}}
    [\mathbf{x}]_{\mathcal{C}}
    =
    \mathbf{x}
\]%

So we have a way to use one as a basis and transform between them.

Hence if $ \mathcal{C} $ is a basis, then $ \mathcal{B} $ must also be.

\subsection{Part (b)} 


\[%
    \mathbf{x} 
    =
    5 \mathbf{b}_{1} + 3 \mathbf{b}_{2} 
    =
    \mathbf{P}_{\mathcal{B}}
    \begin{bmatrix} 5 \\ 3 \end{bmatrix}
\]%

\[%
    [\mathbf{x}]_{\mathcal{C}} 
    =
    \mathbf{P}_{\mathcal{C}}^{-1}
    \mathbf{P}_{\mathcal{B}}
    [\mathbf{x}]_{\mathcal{B}}  
    = 
    \mathbf{P}_{\mathcal{C}\mapsfrom \mathcal{B}} 
    [\mathbf{x}]_{\mathcal{B}}
\]%

From the previous question we have:
\[%
    \mathbf{P}_{\mathcal{C}\mapsfrom \mathcal{B}} 
    =
    \begin{bmatrix} \ -1 & 2\\ 3 & -4 \ \end{bmatrix}
\]%

Hence:
\[%
    [\mathbf{x}]_{\mathcal{C}}
    =
    \begin{bmatrix} \ -1 & 2\\ 3 & -4 \ \end{bmatrix}
    \begin{bmatrix} 5 \\ 3 \end{bmatrix}
    =
    \left[\begin{matrix}1\\3\end{matrix}\right]
\]%

\subsection{Part (c)} 


\[%
    \mathbf{y}
    =
    3 \mathbf{c}_{1} - 5 \mathbf{c}_{2}
    = 
    \mathbf{P}_{\mathcal{C}}
    \begin{bmatrix} 3 \\ -5 \end{bmatrix}
    =
    \mathbf{P}_{\mathcal{C}}
    [\mathbf{y}]_{\mathcal{C}}
\]%

\[%
    [\mathbf{y}]_{\mathcal{B}} 
    = 
    \mathbf{P}_{\mathcal{B\mapsfrom C}}
    \mathbf{P}_{\mathcal{C}}
    [\mathbf{y}]_{\mathcal{C}}
    =
    (\mathbf{P}_{\mathcal{C\mapsfrom B}})^{-1}
    \mathbf{P}_{\mathcal{C}}
    [\mathbf{y}]_{\mathcal{C}}
\]%

From the previous question we have:
\[%
    \mathbf{P}_{\mathcal{C}\mapsfrom \mathcal{B}} 
    =
    \begin{bmatrix} \ -1 & 2\\ 3 & -4 \ \end{bmatrix}
\]%

Hence:

\[%
    [\mathbf{y}]_{\mathcal{B}}
    =
    (\mathbf{P}_{\mathcal{C\mapsfrom B}})^{-1}
    \mathbf{P}_{\mathcal{C}}
    [\mathbf{y}]_{\mathcal{C}}
    =
    \begin{bmatrix} \ -1 & 2\\ 3 & -4 \ \end{bmatrix}^{-1}
    \begin{bmatrix} -1 & 4 \\ -1 & 1 \\\end{bmatrix}
    \begin{bmatrix} 3 \\ -5 \end{bmatrix}
    =
    \begin{bmatrix} -54 \vspace{1mm}\\  -\frac{77}{2}  \end{bmatrix}
\]%

\clearpage
\section{Question 4} 
\subsection{Part (a)} 

\[%
    \mathcal{Q}
    =
    \begin{bmatrix} \
        \mathbf{q}_{1} \
        \mathbf{q}_{2} \
        \mathbf{q}_{3} \ 
        \mathbf{q}_{4} \
    \end{bmatrix}
    =
    \begin{bmatrix} 
        0 & 0 & 0 & 1 \\
		0 & 0 & 1 & 1 \\
		0 & 1 & 1 & 1 \\
		1 & 1 & 1 & 1 \\		
    \end{bmatrix}
\]%


\[%
    \text{rref} (\mathcal{Q}) 
    =
    \begin{bmatrix} 
        1 & 0 & 0 & 0 \\
		0 & 1 & 0 & 0 \\
		0 & 0 & 1 & 0 \\
		0 & 0 & 0 & 1 \\		
    \end{bmatrix}
\]%

$ \mathcal{Q} $ has 4 pivots, meaning it must span $ \R^{4} $ and be linearly
independent, meeting the criteria for a basis.

\subsection{Part (b)} 

\[%
    \mathbf{P}_{\mathcal{Q}\mapsfrom \mathcal{H}}
    =
    \mathbf{P}_{\mathcal{Q}}^{-1} \mathbf{P}_{\mathcal{H}}
    =
    \begin{bmatrix} 
        0 & 0 & 0 & 1 \\
		0 & 0 & 1 & 1 \\
		0 & 1 & 1 & 1 \\
		1 & 1 & 1 & 1 \\		
    \end{bmatrix}^{-1}
    \begin{bmatrix} 
        0 & 0 & 0 & 8 \\
		0 & 0 & 4 & 0 \\
		0 & 2 & 0 & -12 \\
		1 & 0 & -2 & 0 \\		
    \end{bmatrix}
    =
    \begin{bmatrix} 
        1 & -2 & -2 & 12 \\
		0 & 2  & -4 & -12 \\
		0 & 0  & 4  & -8 \\
		0 & 0  & 0  & 8 \\		
    \end{bmatrix}
\]%

\subsection{Part (c)} 

\[%
    [\mathbf{r}]_{\mathcal{H}} 
    =
    \mathbf{P}_{\mathcal{H}}^{-1}\mathbf{r} 
\]%


\[%
    =
    \begin{bmatrix} 
        0 & 0 & 0 & 8 \\
		0 & 0 & 4 & 0 \\
		0 & 2 & 0 & -12 \\
		1 & 0 & -2 & 0 \\		
    \end{bmatrix}^{-1}  
    \begin{bmatrix} 3\\2\\1\\0 \end{bmatrix}
    =
    \begin{bmatrix} 
        1\\
        \frac{11}{4} \vspace{1.5mm} \\
        \frac{1}{2}  \vspace{1.5mm} \\ 
        \frac{3}{8}  
    \end{bmatrix}
\]%

\subsection{Part (d)} 

\[%
    [\mathbf{r}]_{\mathcal{Q}}
    =
    \mathbf{P}_{\mathcal{Q}\mapsfrom \mathcal{H}}
    [\mathbf{r}]_{\mathcal{H}}
    =
    \begin{bmatrix} 
        1 & -2 & -2 & 12 \\
		0 & 2  & -4 & -12 \\
		0 & 0  & 4  & -8 \\
		0 & 0  & 0  & 8 \\		
    \end{bmatrix}
    \begin{bmatrix} 
        1\\
        \frac{11}{4} \vspace{1.5mm} \\
        \frac{1}{2}  \vspace{1.5mm} \\ 
        \frac{3}{8}  
    \end{bmatrix}
    =
    \begin{bmatrix} -1\\-1\\-1\\3 \end{bmatrix}
\]%

 
 




\clearpage



-

\end{document}

