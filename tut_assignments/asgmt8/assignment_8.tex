\documentclass{article}
\author{Alex Hiller (11850637)}
\title{Linear Algebra, Assignment 8}
% Type-setting
\setlength{\parindent}{0cm}
\setlength{\parskip}{0.125cm}
%\pagenumbering{gobble}
\usepackage[margin=2.5cm]{geometry} % Formatting
\usepackage{amsmath}      % Mathematics
\usepackage{amssymb}      % Mathematics
\usepackage{listings}     % Listings
\usepackage{esint}        % Mathematics
\usepackage{color}        % Listings
\usepackage{courier}      % Listings
\usepackage{circuitikz}   % Circuits
\usepackage{titlesec}     % Section Formatting
\usepackage{stmaryrd}     % \mapsfrom arrow. 
\usepackage{mathtools}    % \coloneqq
\usepackage{svg}
\usepackage{multicol}
\columnsep=75pt
\usepackage{blindtext}
\setsvg{inkscape=inkscape -z -D}
\input{/home/polluticorn/GitHub/texTemplates/texMacros}
% Section formatting
\titleformat{\section}{\huge \bfseries}{}{0em}{}[]
\titleformat{\subsection}{\Large \bfseries}{}{0em}{}
\titleformat{\subsubsection}{\bfseries}{}{0em}{}

%%%%%%%%%%%%%%%%%%%%%%%%%%%%%%%%%%%%%%%%%%%%%%%%%%%%%%%%%%%%%%%%%%%%%%%%%%%%%%%%

\begin{document}
\maketitle
\tableofcontents
\clearpage
\section{Question 1} 
\subsection{Part (a)} 
\[%
    \mathbf{y}
    =
    \begin{bmatrix} 7\\1 \end{bmatrix}
    \qquad
    \mathbf{u}
    =
    \begin{bmatrix} 8\\-6 \end{bmatrix}
\]%

Finding $\mathbf{y}=\mathbf{\hat{y}+\mathbf{z}}$

\[%
    \mathbf{\hat{y}} 
    =
    \text{proj}_{\mathbf{u}}\left(\mathbf{y}\right)
    =
    \sum_{i=1}^{p} \frac{\mathbf{y} \cdot \mathbf{u_i} }{\mathbf{u_i} \cdot
    \mathbf{u_i}}  \cdot \mathbf{u_i}
    \quad
    \Rightarrow 
    \quad
    p=1
    \quad 
    \Rightarrow 
    \quad 
    \mathbf{\hat{y}}
    =
    \frac{\mathbf{y} \cdot \mathbf{u_1}}{\mathbf{u_1}  \cdot \mathbf{u_1}}
    \cdot \mathbf{u_1}
\]%

\[%
    \mathbf{y} \cdot \mathbf{u_1} 
    =
    (7)(8)+(1)(-6)
    =
    56-6
    =
    50
\]%


\[%
    \mathbf{u_1} \cdot \mathbf{u_1}
    =
    (8)(8)+(-6)(-6)
    =
    64+36
    =
    100
\]%

\[%
    \mathbf{\hat{y}}
    =
    \frac{50}{100} 
    \begin{bmatrix} 8\\-6 \end{bmatrix}
    =
    \frac{1}{2} 
    \begin{bmatrix} 8\\-6 \end{bmatrix}
\]%

\[%
    \therefore \
    \mathbf{\hat{y}}=
    \begin{bmatrix} 4\\-3 \end{bmatrix}
\]%


\[%
    \mathbf{z}
    =
    \mathbf{y}
    -
    \mathbf{\hat{y}}
\]%

\[%
     \therefore 
     \mathbf{z}
     =
     \begin{bmatrix} 7\\1 \end{bmatrix}
     -
     \begin{bmatrix} 4\\-3 \end{bmatrix}
     =
     \begin{bmatrix} 3\\4 \end{bmatrix}
\]%

Hence the orthogonal decomposition is:

\[%
    \mathbf{y}=
    \begin{bmatrix} 4\\-3 \end{bmatrix}
    +
    \begin{bmatrix} 3\\4 \end{bmatrix}
\]%

\subsection{Part (b)} 
This would just be the modulus of $\mathbf{z}$.

\[%
    \|\mathbf{z}\|
    =
    \sqrt{3^{2}+4^{2}}
    =
    \sqrt{9+16}
    =
    5
\]%

\clearpage
\section{Question 2} 


\[%
    \mathcal{W} = \text{Span} \left\{ \mathbf{u_1}, \mathbf{u_2} \right\}
\]%


\[%
    \mathbf{y_1}
    =
    \begin{bmatrix} 5\\3\\5 \end{bmatrix}
    \qquad
    \mathbf{y_2}
    =
    \begin{bmatrix} 3\\-2\\ 0 \end{bmatrix}
    \qquad
    \mathbf{u_1}
    =
    \begin{bmatrix}0\\2\\-1\end{bmatrix}
    \qquad
    \mathbf{u_2}
    =
    \begin{bmatrix}-15\\2\\4\end{bmatrix}
\]%

\subsection{Part (a)} 

First vector:
\[%
    \mathbf{y_1}
    =
    \mathbf{\hat{y}_{1}}+\mathbf{z_1}
\]%


\[%
    \mathbf{\hat{y_1}}
    =
    \text{proj}_{\mathcal{W}}\left(\mathbf{y_1}\right)
    =
    \sum_{i=1}^{p=2} 
    \frac{\mathbf{y_1} \cdot \mathbf{u_i} }{\mathbf{u_i} \cdot
    \mathbf{u_i}}  \cdot \mathbf{u_i}
\]%



\[%
    \mathbf{\hat{y_1}}
    =
    \frac{\mathbf{y_1} \cdot \mathbf{u_1} }{\mathbf{u_1} \cdot
    \mathbf{u_1}}  \cdot \mathbf{u_1}
    +
    \frac{\mathbf{y_1} \cdot \mathbf{u_2} }{\mathbf{u_2} \cdot
    \mathbf{u_2}}  \cdot \mathbf{u_2}
    =
    \begin{bmatrix}3\\0\\-1\end{bmatrix}
\]%


\[%
    \mathbf{z_1}
    =
    \mathbf{y_1}-\mathbf{\hat{y_1}}
    =
    \begin{bmatrix}5\\3\\5\end{bmatrix}
    -
    \begin{bmatrix}3\\0\\-1\end{bmatrix}
    =
    \begin{bmatrix}2\\3\\6\end{bmatrix}
\]%

\[%
     \therefore \mathbf{y_1} = 
     \begin{bmatrix}3\\0\\-1\end{bmatrix}
     +
     \begin{bmatrix}2\\3\\6\end{bmatrix}
\]%


Second vector:
\[%
    \mathbf{y_2}
    =
    \mathbf{\hat{y_2}}
    +
    \mathbf{z_2}
\]%

\[%
    \mathbf{\hat{y_2}}
    =
    \frac{\mathbf{y_2} \cdot \mathbf{u_1} }{\mathbf{u_1} \cdot
    \mathbf{u_1}}  \cdot \mathbf{u_1}
    +
    \frac{\mathbf{y_2} \cdot \mathbf{u_2} }{\mathbf{u_2} \cdot
    \mathbf{u_2}}  \cdot \mathbf{u_2}
    =
    \begin{bmatrix}3\\-2\\0\end{bmatrix}
\]%


\[%
    \mathbf{z_2}
    =
    \mathbf{y_2}-\mathbf{\hat{y_2}}
    =
    \begin{bmatrix}3\\-2\\0\end{bmatrix}
    -
    \begin{bmatrix}3\\-2\\0\end{bmatrix}
    =
    0
\]%

\[%
     \therefore \mathbf{y_2} = 
     \begin{bmatrix}3\\-2\\0\end{bmatrix}
     +
     \begin{bmatrix}0\\0\\0\end{bmatrix}
\]%

\subsection{Part (b)} 
Distance from $\mathbf{y_1}$ to $\mathcal{W}$ is equal to the modulus of the $\mathbf{z}$
component of the orthogonal decomposition.

 

\[%
    \|\mathbf{z_1}\|
    =
    \sqrt{2^{2}+3^{2}+6^{2}}
    =
    7
\]%

\[%
    \|\mathbf{z_2}\|
    =
    0
\]%

Therefore, $\mathbf{y_1}$ is 7 units from $\mathcal{W}$ and $\mathbf{y_2}$ is 0
units from $\mathcal{W}$. 

\[%
     \therefore \mathbf{y_2}
     \in 
     \mathcal{W}
\]%


\clearpage
\section{Question 3} 
\subsection{Part (a)} 
The Gram-Schmidt (a method used to convert a basis to an orthogonal basis) can concisely be written as:

\[%
    \mathbf{v_k}
    =
    \mathbf{x_k}
    -
    \sum_{j=1}^{k-1} \text{proj}_{\mathbf{v_j}}\left(\mathbf{x_k}\right)
\]%

Our vectors are:

\[%
    \mathbf{a_1}
    =
    \begin{bmatrix}1\\-1\\1\\-1\end{bmatrix}
    \quad
    \mathbf{a_2}
    =
    \begin{bmatrix}4\\2\\2\\0\end{bmatrix}
    \quad
    \mathbf{a_3}
    =
    \begin{bmatrix}4\\3\\2\\1\end{bmatrix}
\]%

Finding our orthogonal set \ldots


\[%
    \mathbf{v_1}
    =
    \mathbf{x_1}
\]%

\[%
    \mathbf{v_2}
    =
    \mathbf{x_2}
    -
    \text{proj}_{\mathbf{v_1}}\left(\mathbf{x_2}\right)
    =
    \mathbf{v_1}
    -
    \frac{\mathbf{x_2}\cdot \mathbf{v_1}}{\mathbf{v_1}\cdot \mathbf{v_1}} \mathbf{v_1}
\]%

\[%
    \mathbf{v_3}
    =
    \mathbf{x_3}
    -
    \text{proj}_{\mathbf{v_1}}\left(\mathbf{x_3}\right)
    -
    \text{proj}_{\mathbf{v_2}}\left(\mathbf{x_3}\right)
    =
    \mathbf{v_1}
    -
    \frac{\mathbf{x_2}\cdot \mathbf{v_1}}{\mathbf{v_1}\cdot
    \mathbf{v_1}}\mathbf{v_1}
    -
    \frac{\mathbf{x_3}\cdot \mathbf{v_2}}{\mathbf{v_2}\cdot \mathbf{v_2}}\mathbf{v_2}
\]%

\[%
    \mathbf{v_1}
    =
    \begin{bmatrix}1\\-1\\1\\-1\\\end{bmatrix}
    \quad
    \mathbf{v_2}
    =
    \begin{bmatrix}3\\3\\1\\1\end{bmatrix}
    \quad
    \mathbf{v_3}
    =
    \begin{bmatrix}-{1}/{10} \\-{1}/{10} \\{3}/{10} \\{3}/{10} \end{bmatrix}
\]%

\subsection{Part (b)} 
Finding our orthonormal set:


\[%
    \mathbf{u_1}
    =
    \frac{\mathbf{v_1}}{\|\mathbf{v_1}\|}
    \qquad
    \mathbf{u_2}
    =
    \frac{\mathbf{v_2}}{\|\mathbf{v_2}\|}
    \qquad
    \mathbf{u_3}
    =
    \frac{\mathbf{v_3}}{\|\mathbf{v_3}\|}
\]%

\[%
    \mathbf{u_1}
    =
    \begin{bmatrix}1/2\\-1/2\\1/2\\-1/2\end{bmatrix}
    \qquad
    \mathbf{u_2}
    =
    \begin{bmatrix}3\sqrt{5}/10\\3\sqrt{5}/10\\\sqrt{5}/10\\\sqrt{5}/10\end{bmatrix}
    \qquad
    \mathbf{u_3}
    =
    \begin{bmatrix}-\sqrt{5}/10\\-\sqrt{5}/10\\3\sqrt{5}/10\\3\sqrt{5}/10\end{bmatrix}    
\]%

\clearpage
\section{Question 4} 
Ways to check whether a vector belongs to certain spaces:

\[%
    \mathbf{v} \in  \text{Col}(\mathbf{A}) \text{ if } 
    \mathbf{A}\mathbf{x}
    =
    \mathbf{v}
\]%

\[%
    \mathbf{v} \in \text{Row}(\mathbf{A}) \text{ if } 
    \mathbf{A}^{T}\mathbf{x}
    =
    \mathbf{v}
\]%

\[%
    \mathbf{v} \in \text{Nul}(\mathbf{A}) \text{ if } 
    \mathbf{A}\mathbf{v}
    =
    \mathbf{0}
\]%
\subsection{Part (a)} 
Can a vector $\mathbf{v}$ exist s.t. $\mathbf{v} \in \text{Col}(\mathbf{A})$ and
$\mathbf{v} \in \text{Row}(\mathbf{A})$?

If $\mathbf{v} \in \text{Col}(\mathbf{A})$, then:

\[%
    \mathbf{A}\mathbf{x}=\mathbf{v}
\]%

But if $\mathbf{v} \in \text{Row}(\mathbf{A})$, then:

\[%
    \mathbf{A}^{T}\mathbf{x}
    =
    \mathbf{v}
\]%

Then:

\[%
    \mathbf{A}^{-1}\mathbf{A}^{T}\mathbf{x}=\mathbf{x}
\]%

Meaning the following must hold for $\mathbf{v}$ to be contained in both vector
spaces.

\[%
    \mathbf{A}^{-1}\mathbf{A}^{T} = \mathbf{I}
\]%

\subsection{Part (b)} 
Can a vector $\mathbf{v}$ exist s.t. $\mathbf{v} \in \text{Row}(\mathbf{A})$ and
$\mathbf{v} \in \text{Nul}(\mathbf{A})$?


\subsection{Part (c)} 
Can a vector $\mathbf{v}$ exist s.t. $\mathbf{v} \in \text{Col}(\mathbf{A})$ and
$\mathbf{v} \in \text{Nul}(\mathbf{A})$?




\end{document}

