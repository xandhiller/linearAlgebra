\documentclass{article}
\author{Alex Hiller (11850637)}
\title{Linear Algebra, Assignment 7}
% Type-setting
\setlength{\parindent}{0cm}
\setlength{\parskip}{0.125cm}
%\pagenumbering{gobble}
\usepackage[margin=2.5cm]{geometry} % Formatting
\usepackage{amsmath}      % Mathematics
\usepackage{amssymb}      % Mathematics
\usepackage{listings}     % Listings
\usepackage{esint}        % Mathematics
\usepackage{color}        % Listings
\usepackage{courier}      % Listings
\usepackage{circuitikz}   % Circuits
\usepackage{titlesec}     % Section Formatting
\usepackage{stmaryrd}     % \mapsfrom arrow. 
\usepackage{mathtools}    % \coloneqq
\usepackage{svg}
\setsvg{inkscape=inkscape -z -D}
\input{/home/polluticorn/GitHub/texTemplates/texMacros}
% Section formatting
\titleformat{\section}{\huge \bfseries}{}{0em}{}[]
\titleformat{\subsection}{\Large \bfseries}{}{0em}{}
\titleformat{\subsubsection}{\bfseries}{}{0em}{}

%%%%%%%%%%%%%%%%%%%%%%%%%%%%%%%%%%%%%%%%%%%%%%%%%%%%%%%%%%%%%%%%%%%%%%%%%%%%%%%%

\begin{document}
\maketitle
\tableofcontents
\clearpage

\section{Question 1} 
\subsection{Question 1: Part 1 (a)} 

\[%
    \mathbf{A} 
    = 
    \begin{bmatrix} 
        1 & 7 & -6 & -1 & -3 \\
		1 & -8 & 9 & -3 & 10 \\
		-2 & -9 & 7 & 1 & 0\\		
    \end{bmatrix}
\]%
\\
\[%
    \text{ref} (\mathbf{A}) 
    =
    \begin{bmatrix} \vspace{2mm}
        2 & \frac{9}{2}  & -\frac{7}{2}  & -\frac{1}{2}  & 0 \\ \vspace{2mm}
		0 & 5 & -5 & 1 & -4 \\
		0 & 0 & 0 & 1 & 1 \\		
    \end{bmatrix}
\]%

\[%
    \text{rref} (\mathbf{A}) 
    = 
    \begin{bmatrix} 
        1 & 0 & 1 & 0 & 5 \\
		0 & 1 & -1 & 0 & -1 \\
		0 & 0 & 0 & 1 & 1 \\		
    \end{bmatrix}
\]%

\subsubsection{Col(\textbf{A})} 

\[%
    \text{Span} \left\{
    \begin{bmatrix} 1\\1\\2\\ \end{bmatrix},
    \begin{bmatrix} 7\\-8\\-9\\ \end{bmatrix},
    \begin{bmatrix} -1\\-3\\1\\ \end{bmatrix}
    \right\}
    =
    \text{Span} \left\{ 
    \mathbf{u},\mathbf{v},\mathbf{w}\right\}
    \Rightarrow 
    \text{Col}(\mathbf{A}) = 
    \left\{  
    a \cdot \mathbf{u} +
    b \cdot \mathbf{v} +
    c \cdot \mathbf{w} 
    =
    \mathbf{0}
    \mid
    \left\{ a,b,c \right\} \in \R
    \right\}
\]%


\subsubsection{Nul(\textbf{A})} 
Free variables:
\[%
    \left\{ x_3,x_5 \right\}
\]%

Basic variables:

\[%
    \left\{ x_1,x_2,x_4 \right\}
\]%

Expressing basic in terms of free:

\[%
    \begin{bmatrix} 
        x_1 \\
		x_2 \\
		x_3 \\
		x_4 \\
		x_5 \\		
    \end{bmatrix}
    =
    \begin{bmatrix} 
        -x_3 -5 x_5 \\
		x_3+x_5 \\
		x_3 \\
		-x_5 \\
		x_5 \\		
    \end{bmatrix}
    =
    x_3
    \begin{bmatrix} 
        -1 \\
		1 \\
		1 \\
		0 \\
		0 \\		
    \end{bmatrix}
    +
    x_5
    \begin{bmatrix} 
        -5 \\
		1 \\
		0 \\
		-1 \\
		1 \\		
    \end{bmatrix}
\]%

So Nul($ \mathbf{A} $) is

\[%
    \text{Span} \left\{  
    \begin{bmatrix} 
        -1 \\
		1 \\
		1 \\
		0 \\
		0 \\		
    \end{bmatrix}
    ,
    \begin{bmatrix} 
        -5 \\
		1 \\
		0 \\
		-1 \\
		1 \\		
    \end{bmatrix}
    \right\}
    =
    \text{Span} \left\{ \mathbf{u_{n},} \mathbf{v_n}  \right\}
    \Rightarrow 
    \text{Nul}(\mathbf{A})
    =
    \left\{ 
    a \cdot\mathbf{u_{n}} + b \cdot \mathbf{v_n} = \mathbf{0}  \mid \left\{ a,b \right\}
    \in \R
    \right\}
\]%



\subsubsection{Row(\textbf{A})} 

\[%
    \text{Span} \left\{
    \begin{bmatrix} \vspace{2mm}
        2 \\ \vspace{2mm}
        \frac{9}{2}  \\ \vspace{2mm}
        -\frac{7}{2}  \\ \vspace{2mm}
        -\frac{1}{2}  \\ \vspace{2mm}
        0 \\ 
    \end{bmatrix},
    \begin{bmatrix} \vspace{2mm}
        0  \\ \vspace{2mm}
        5  \\ \vspace{2mm}
        -5 \\ \vspace{2mm}
        1  \\ \vspace{2mm}
        -4 \\ 
    \end{bmatrix},
    \begin{bmatrix} \vspace{2mm}
        \ 0 \\ \vspace{2mm} 
        \ 0 \\ \vspace{2mm} 
        \ 0 \\ \vspace{2mm} 
        \ 1 \\ \vspace{2mm} 
        \ 1 \\		
    \end{bmatrix}
    \right\}
    =
    \text{Span} \left\{ 
    \mathbf{u_{r}},
    \mathbf{v_{r}},
    \mathbf{w_{r}},
    \right\}
    \Rightarrow 
    \text{Row}(\mathbf{A})=
    \left\{
    a \cdot \mathbf{u_{r}} +
    b \cdot \mathbf{v_{r}} +
    c \cdot \mathbf{w_{r}}
    =
    \mathbf{0}
    \mid   
    \left\{ a,b,c \right\} \in \R
    \right\}
\]%

\subsection{Question 1: Part 1 (b)} 
Basis for Col($ \mathbf{A} $)

\[%
    \left\{
    \begin{bmatrix} 1\\1\\2\\ \end{bmatrix},
    \begin{bmatrix} 7\\-8\\-9\\ \end{bmatrix},
    \begin{bmatrix} -1\\-3\\1\\ \end{bmatrix}
    \right\}
\]%


Basis for Nul($ \mathbf{A} $)

\[%
    \left\{     
    \begin{bmatrix} 
        -1 \\
		1 \\
		1 \\
		0 \\
		0 \\		
    \end{bmatrix}
    ,
    \begin{bmatrix} 
        -5 \\
		1 \\
		0 \\
		-1 \\
		1 \\		
    \end{bmatrix}
    \right\}
\]%


Basis for Row($ \mathbf{A} $)

\[%
    \left\{
    \begin{bmatrix} \vspace{2mm}
        2 \\ \vspace{2mm}
        \frac{9}{2}  \\ \vspace{2mm}
        -\frac{7}{2}  \\ \vspace{2mm}
        -\frac{1}{2}  \\ \vspace{2mm}
        0 \\ 
    \end{bmatrix},
    \begin{bmatrix} \vspace{2mm}
        0  \\ \vspace{2mm}
        5  \\ \vspace{2mm}
        -5 \\ \vspace{2mm}
        1  \\ \vspace{2mm}
        -4 \\ 
    \end{bmatrix},
    \begin{bmatrix} \vspace{2mm}
        \ 0 \\ \vspace{2mm} 
        \ 0 \\ \vspace{2mm} 
        \ 0 \\ \vspace{2mm} 
        \ 1 \\ \vspace{2mm} 
        \ 1 \\		
    \end{bmatrix}
    \right\}
\]%



\subsection{Question 1: Part 1 (c)} 
dim(Col($\mathbf{A}$)) = 3

dim(Nul($ \mathbf{A} $)) = 2

dim(Row($ \mathbf{A} $)) = dim(Col($\mathbf{A}$)) = 3

\clearpage
\subsection{Question 1: Part 1 (d)} 

\[%
    x 
    =
    \begin{bmatrix} 2\\-9\\3 \end{bmatrix}
    \quad
    y
    =
    \begin{bmatrix} 4\\0\\1\\1\\-1 \end{bmatrix}
\]%

\subsubsection{$ x \in \text{Col}(\mathbf{A}) $ ?} 

Take the $\text{Col}(\mathbf{A})$, form a matrix, invert and multiply by $x$. 

\[%
    \begin{bmatrix} 
        1&1&2\\ 
        7&-8&-9\\ 
        -1&-3&1\\ 
    \end{bmatrix}^{-1}
    \begin{bmatrix} 2\\-9\\3 \end{bmatrix}
    =
    \begin{bmatrix} \frac{25}{13} \vspace{2mm}\\ \frac{5}{13} \vspace{2mm} \\ \frac{34}{13}  \end{bmatrix}
\]%

$$\therefore x \in \text{Col}(\mathbf{A})$$

\subsubsection{$ x \in \text{Nul}(\mathbf{A}) \ ?$}
The dimensions of the Null space do not match $x$ and so it \textbf{cannot} be in the
Null space. i.e. the Null space is made of vectors that have 5 entries, where as 
$x$ only has 3.

\subsubsection{$ x \in \text{Row}(\mathbf{A}) \ ?$}
The dimensions of the Row space do not match $x$ and so it \textbf{cannot} be in the
Row space. i.e. the Row space is made of vectors that have 5 entries, where as
$x$ only has 3.

\vspace{2mm}

\subsubsection{$ y \in \text{Col}(\mathbf{A}) \ ?$}
The dimensions of the Column space does not match $y$ and so it \textbf{cannot} be in the
Column space. i.e. the Column space is made of vectors that have 3 entries, where as
$y$ has 5.

\subsubsection{$ y \in \text{Nul}(\mathbf{A}) \ ?$}


\[%
    a \mathbf{u_n} + b \mathbf{v_n} = \mathbf{y}
\]%
\[%
    a
    \begin{bmatrix} 
        -1 \\
		1 \\
		1 \\
		0 \\
		0 \\		
    \end{bmatrix}
    +
    b
    \begin{bmatrix} 
        -5 \\
		1 \\
		0 \\
		-1 \\
		1 \\		
    \end{bmatrix}
    =
    \begin{bmatrix} 4\\0\\1\\1\\-1 \end{bmatrix}
\]%

Form an augmented matrix:
\[%
    \mathcal{N} = 
    \begin{bmatrix} 
        -1 & -5 & 4 \\
		1 & 1 & 0 \\
		1 & 0 & 1 \\
		0 & -1 & 1 \\
		0 & 1 & -1 \\		
    \end{bmatrix}
\]%

Get into reduced echelon form:

\[%
    \text{rref} (\mathcal{N}) 
    =
    \begin{bmatrix} 
        1 & 0 & 1 \\
		0 & 1 & -1 \\
		0 & 0 & 0 \\
		0 & 0 & 0 \\
		0 & 0 & 0 \\		
    \end{bmatrix}
\]%

So $a = 1$ and $b = -1$, so  $y$ is a linear combination of the null vectors,
meaning it lies in the space Null space.

\[%
    \therefore y \in \text{Nul}(\mathbf{A})
\]%

\subsubsection{$ y \in \text{Row}(\mathbf{A}) \ ?$}

\[%
    a \mathbf{u_r} + b \mathbf{v_r} + c \mathbf{w_r}  = \mathbf{y}
\]%
\[%
    a
     \begin{bmatrix} 
        \vspace{2mm}
        2 \\ \vspace{2mm}
        \frac{9}{2}  \\ \vspace{2mm}
        -\frac{7}{2}  \\ \vspace{2mm}
        -\frac{1}{2}  \\ \vspace{2mm}
        0 \\ 
    \end{bmatrix}
    +
    b
    \begin{bmatrix} \vspace{2mm}
        0  \\ \vspace{2mm}
        5  \\ \vspace{2mm}
        -5 \\ \vspace{2mm}
        1  \\ \vspace{2mm}
        -4 \\ 
    \end{bmatrix}
    +
    c
    \begin{bmatrix} \vspace{2mm}
        \ 0 \\ \vspace{2mm} 
        \ 0 \\ \vspace{2mm} 
        \ 0 \\ \vspace{2mm} 
        \ 1 \\ \vspace{2mm} 
        \ 1 \\		
    \end{bmatrix}
    =
    \begin{bmatrix} 4\\0\\1\\1\\-1 \end{bmatrix}
\]%

Form an augmented matrix:
\[%
    \mathcal{C} = 
    \begin{bmatrix}     \vspace{2mm}
        2            &  0  & \ 0 & 4 \\ \vspace{2mm}
        \frac{9}{2}  &  5  & \ 0 & 0 \\ \vspace{2mm}
        -\frac{7}{2} &  -5 & \ 0 & 1 \\ \vspace{2mm}
        -\frac{1}{2} &  1  & \ 1 & 1 \\ \vspace{2mm}
        0            &  -4 & \ 1 & -1 \\ 
    \end{bmatrix}
\]%

Get into reduced echelon form:

\[%
    \text{rref} (\mathcal{C}) 
    =
    \begin{bmatrix} 
        1 & 0 & 0 & 0 \\
		0 & 1 & 0 & 0 \\
		0 & 0 & 1 & 0 \\
		0 & 0 & 0 & 1 \\
		0 & 0 & 0 & 0 \\		
    \end{bmatrix}
\]%

Here, we have an inconsistent system, meaning that $y \notin
\text{Row}(\mathbf{A})$.

%%%%%%%%%%%%%%%%%%%%%%%%%%%%%%%%%%%%%%%%%%%%%%%%%%%%%%%%%%%%%%%%%%%%%%%%%%%%%%%%
%%%%%%%%%%%%%%%%%%%%%%%%%%%%%%%%%%%%%%%%%%%%%%%%%%%%%%%%%%%%%%%%%%%%%%%%%%%%%%%%
%%%%%%%%%%%%%%%%%%%%%%%%%%%%%%%%%%%%%%%%%%%%%%%%%%%%%%%%%%%%%%%%%%%%%%%%%%%%%%%%
\subsection{Question 1: Part 2 (a)} 


\[%
    \mathbf{A}
    =
    \begin{bmatrix} 
        1 & 2 & 3 \\
		3 & 7 & 8 \\
		0 & -3 & 3 \\
		-1 & 3 & -8 \\		
    \end{bmatrix}
\]%

\[%
    \text{ref} (\mathbf{A}) 
    =
    \begin{bmatrix} 
        3 & 7 & 8 \\
		0 & 1 & -1 \\
		0 & 0 & 0 \\
		0 & 0 & 0 \\		
    \end{bmatrix}
\]%

\[%
    \text{rref} (\mathbf{A}) 
    =
    \begin{bmatrix} 
        1 & 0 & 5 \\
		0 & 1 & -1 \\
		0 & 0 & 0 \\
		0 & 0 & 0 \\		
    \end{bmatrix}
\]%

\subsubsection{Col(\textbf{A})} 

\[%
    \text{Span} \left\{
    \begin{bmatrix} 1\\3\\0\\-1 \end{bmatrix},
    \begin{bmatrix} 2\\7\\-3\\3 \end{bmatrix}
    \right\}
    =
    \text{Span} \left\{ 
    \mathbf{u},\mathbf{v}\right\}
    \Rightarrow 
    \text{Col}(\mathbf{A}) = 
    \left\{  
    a \cdot \mathbf{u} +
    b \cdot \mathbf{v} 
    =
    \mathbf{0}
    \mid
    \left\{ a,b \right\} \in \R
    \right\}
\]%

\subsubsection{Nul(\textbf{A})} 
Free variables
\[%
    \left\{ x_3 \right\}
\]%
Basic variables:
\[%
    \left\{ x_1, x_2 \right\}
\]%


Expressing basic in terms of free:

\[%
    \begin{bmatrix} 
        x_1 \\
		x_2 \\
		x_3 \\
    \end{bmatrix}
    =
    \begin{bmatrix} -5 x_3 \\ x_3 \\ x_3 \end{bmatrix}
    =
    x_3
    \begin{bmatrix} -5\\1\\1 \end{bmatrix}
\]%

\[%
    \text{Span} \left\{  
    \begin{bmatrix} 
        -5\\1\\1
    \end{bmatrix}
    \right\}
    =
    \text{Span} \left\{ \mathbf{u_{n}}   \right\}
    \Rightarrow 
    \text{Nul}(\mathbf{A})
    =
    \left\{ 
    a \cdot\mathbf{u_{n}} = \mathbf{0}  \mid \left\{ a \right\}
    \in \R
    \right\}
\]%

\subsubsection{Row(\textbf{A})} 

\[%
    \text{Span} \left\{
    \begin{bmatrix} 1\\0\\5\\   \end{bmatrix},
    \begin{bmatrix} 0\\1\\-1\\  \end{bmatrix}
    \right\}
    =
    \text{Span} \left\{ 
    \mathbf{u_{r}},
    \mathbf{v_{r}}
    \right\}
    \Rightarrow 
    \text{Row}(\mathbf{A})=
    \left\{
    a \cdot \mathbf{u_{r}} +
    b \cdot \mathbf{v_{r}} 
    =
    \mathbf{0}
    \mid   
    \left\{ a,b \right\} \in \R
    \right\}
\]%


\subsection{Question 1: Part 2 (b)} 
Basis for Col($ \mathbf{A} $)
\[%
    \left\{ 
    \begin{bmatrix} 1\\3\\0\\-1\\ \end{bmatrix},
    \begin{bmatrix} 2\\7\\-3\\3\\ \end{bmatrix}
    \right\}
\]%

Basis for Nul($ \mathbf{A} $)
\[%
    \left\{ 
    \begin{bmatrix} -5\\1\\1 \end{bmatrix}
    \right\}
\]%


Basis for Row($ \mathbf{A} $)
\[%
    \left\{ 
    \begin{bmatrix} 1\\0\\ 5\\ \end{bmatrix},
    \begin{bmatrix} 0\\1\\-1\\ \end{bmatrix}
    \right\}
\]%




\subsection{Question 1: Part 2 (c)} 
dim(Col($ \mathbf{A} $)) = 2

dim(Nul($ \mathbf{A} $)) = 1

dim(Row($ \mathbf{A} $)) = dim(Col($ \mathbf{A} $)) = 2

\subsection{Question 1: Part 2 (d)} 


\[%
    x
    =
    \begin{bmatrix} 1\\2\\3 \end{bmatrix}
    \quad
    y
    =
    \begin{bmatrix} 0\\2\\-6\\10 \end{bmatrix}
\]%


\subsubsection{$ x \in \text{Col}(\mathbf{A}) \ ?$}
The dimensions do not match up.

$x$ has three entries, $\text{Col}(\mathbf{A})$ has four entries, there is no
way that $x$ can be contained in the space spanned by $\text{Col}(\mathbf{A})$.


\[%
    \therefore x \notin \text{Col}(\mathbf{A})
\]%


\subsubsection{$ x \in \text{Nul}(\mathbf{A}) \ ?$}
If $x$ was in $\text{Nul}(\mathbf{A})$, it would have to be a linear combination
of its single basis vector.

\[%
    a \cdot \begin{bmatrix} -5\\1\\1 \end{bmatrix} 
    =
    \begin{bmatrix} 1\\2\\3 \end{bmatrix}
    =
    \begin{bmatrix} -5a\\a\\a \end{bmatrix} 
\]%

This shows us that $a$ would have to simultaneously be equal to 2 and 3, meaning
that:

\[%
    x  \notin \text{Nul}(\mathbf{A})
\]%

\subsubsection{$ x \in \text{Row}(\mathbf{A}) \ ?$}
If $x$ was a linear combination of the basis vectors for the row space, we
would have:


\[%
    a \cdot 
    \begin{bmatrix} 1\\0\\5 \end{bmatrix}
    +
    b  \cdot 
    \begin{bmatrix} 0\\1\\-1 \end{bmatrix}
    =
    \begin{bmatrix} 1\\2\\3 \end{bmatrix}
\]%

Forming an augmented matrix:

\[%
    \begin{bmatrix} 
        1 & 0 & 1 \\
		0 & 1 & 2 \\
		5 & -1 & 3 \\		
    \end{bmatrix}
    \Rightarrow 
    \text{rref} \left(
    \begin{bmatrix} 
        1 & 0 & 1 \\
		0 & 1 & 2 \\
		5 & -1 & 3 \\		
    \end{bmatrix}
    \right)
    =
    \begin{bmatrix} 
        1 & 0 & 2 \\
		0 & 1 & 1 \\
		0 & 0 & 0 \\		
    \end{bmatrix}
\]%

So $a=2$ and $b=1$, meaning that:

\[%
    x \in \text{Row}(\mathbf{A})
\]%


\vspace{2mm}

\subsubsection{$ y \in \text{Col}(\mathbf{A}) \ ?$}
If $y$ is contained $\text{Col}(\mathbf{A})$ then:

\[%
    a \cdot 
    \begin{bmatrix} 1\\3\\0\\-1 \end{bmatrix}
    +
    b \cdot 
    \begin{bmatrix} 2\\7\\-3\\3 \end{bmatrix}
    =
    \begin{bmatrix} 0\\2\\-6\\10 \end{bmatrix} 
\]%

Where $\left\{ a,b \right\}\in \R$.

Forming an augmented matrix with the basis matrix of $\text{Col}(\mathbf{A})$.

\[%
    \begin{bmatrix} 
        1 & 2 & 0 \\
		3 & 7 & 2 \\
		0 & -3 & -6 \\
		-1 & 3 & 10 \\		
    \end{bmatrix}
\]%

\[%
    \text{rref} \left(
    \begin{bmatrix} 
        1 & 2 & 0 \\
		3 & 7 & 2 \\
		0 & -3 & -6 \\
		-1 & 3 & 10 \\		
    \end{bmatrix}
    \right) 
    =
    \begin{bmatrix} 
        1 & 0 & -4 \\
		0 & 1 & 2 \\
		0 & 0 & 0 \\
		0 & 0 & 0 \\		
    \end{bmatrix}
\]%

Hence:

\[%
    y
    \in 
    \text{Col}(\mathbf{A})
    \quad
    \because
    \quad
    y
    =
    -4 \cdot 
    \begin{bmatrix} 1\\3\\0\\-1 \end{bmatrix}
    +
    2 \cdot 
    \begin{bmatrix} 2\\7\\-3\\3 \end{bmatrix}
    =
    \begin{bmatrix} 0\\2\\-6\\10 \end{bmatrix} 

\]%


\subsubsection{$ y \in \text{Nul}(\mathbf{A}) \ ?$}
Dimensions not appropriate.

\[%
    y  \notin \text{Nul}(\mathbf{A})
\]%

\subsubsection{$ y \in \text{Row}(\mathbf{A}) \ ?$}
Dimensions not appropriate.

\[%
    y  \notin \text{Row}(\mathbf{A})
\]%

%%%%%%%%%%%%%%%%%%%%%%%%%%%%%%%%%%%%%%%%%%%%%%%%%%%%%%%%%%%%%%%%%%%%%%%%%%%%%%%%
%%%%%%%%%%%%%%%%%%%%%%%%%%%%%%%%%%%%%%%%%%%%%%%%%%%%%%%%%%%%%%%%%%%%%%%%%%%%%%%%
%%%%%%%%%%%%%%%%%%%%%%%%%%%%%%%%%%%%%%%%%%%%%%%%%%%%%%%%%%%%%%%%%%%%%%%%%%%%%%%%
\section{Question 2} 

\subsection{Part (a)} 
Matrix is 7 equations (rows) by 8 variables (columns), meaning the size is
$(7 \times 8)$.

\[%
    \begin{bmatrix} 
        * & * & * & * & * & * & * & * \\
        * & * & * & * & * & * & * & * \\
        * & * & * & * & * & * & * & * \\
        * & * & * & * & * & * & * & * \\
        * & * & * & * & * & * & * & * \\
        * & * & * & * & * & * & * & * \\
        * & * & * & * & * & * & * & * \\
    \end{bmatrix}
\]%

If two of the rows are linearly dependent, then there must be rank 6.

We can also prove this with the rank theorem:

\[%
    \text{Rank(}\mathbf{A}) + \text{dim(Nul(}\mathbf{A})) = n
\]%

Where $ n \coloneqq  $ number of columns
\[%
    \therefore n=8
\]%
and with there being two dependent equations:
\[%
   \text{dim(Nul(}\mathbf{A})) = 2
\]%
so
\[%
    \text{Rank(}\mathbf{A})  = n  - \text{dim(Nul(}\mathbf{A})) = 8 - 2 
\]%
\\
\[%
    \therefore \text{Rank(}\mathbf{A}) = 6
\]%

\subsection{Part (b)} 
\[%
    \text{dim} (\text{Nul}(\mathbf{A})) = 2
\]%


\subsection{Part (c)} 
No, it does not have a solution for any $\mathbf{b}$. $\mathbf{b}$ would be of
size $(7 \times 1)$. However, the column space of $\mathbf{A}$ only spans up to
$\R^{6}$. Because of this, $\mathbf{A}\mathbf{x} = \mathbf{b}$ cannot express
any $\mathbf{b}$. For that, $\text{Col}(\mathbf{A})$ would have to span
$\R^{7}$.

\subsection{Part (d)} 
By the dimensions of the matrix.

Domain = $ \R^{8} $ 

Codomain = $ \R^{7} $

\subsection{Part (e)} 
Becuase of the two dependent rows, the range can only be spanning up to $ \R^{6} $.



\end{document}

