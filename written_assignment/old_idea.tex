\documentclass{article}
\author{Alex Hiller}
\title{}
% Type-setting
\setlength{\parindent}{0cm}
\setlength{\parskip}{0.125cm}
\pagenumbering{gobble}
\usepackage[margin=2.5cm]{geometry} % Formatting
\usepackage{amsmath}      % Mathematics
\usepackage{amssymb}      % Mathematics
\usepackage{listings}     % Listings
\usepackage{esint}        % Mathematics
\usepackage{color}        % Listings
\usepackage{courier}      % Listings
\usepackage{circuitikz}   % Circuits
\usepackage{titlesec}     % Section Formatting
\usepackage{stmaryrd}               % \mapsfrom arrow. 
\usepackage{svg}
\setsvg{inkscape=inkscape -z -D}
\input{/home/polluticorn/GitHub/texTemplates/texMacros}
% Section formatting
\titleformat{\section}{\huge \bfseries}{}{0em}{}[]
\titleformat{\subsection}{\Large \bfseries}{}{0em}{}
\titleformat{\subsubsection}{\bfseries}{}{0em}{}

%%%%%%%%%%%%%%%%%%%%%%%%%%%%%%%%%%%%%%%%%%%%%%%%%%%%%%%%%%%%%%%%%%%%%%%%%%%%%%%%

\begin{document}

\maketitle

%%% Old idea. %%%%%
\section{2D Geometry} 
The way in which we describe  $ 2D $ Geometry is dependent on something called
homogenous coordinates.

You can describe homogeneous coordinates with:
\[%
    \mathbf{x} = 
    \begin{bmatrix}
        x \\		
        y \\
        1 \\
    \end{bmatrix}
\]%

Expressing a conic as a symmetric matrix: $ \mathbf{A}_{Q} $  is defined as the
matrix required for this expression:

\[%
    \mathbf{x}^{T}\mathbf{A}_{Q} \ \mathbf{x} = 0
\]%
Where:

\[%
    \mathbf{A_{Q} }
    = 
    \begin{bmatrix}\vspace{2mm}
        A            & \frac{B}{2}  & \frac{D}{2}  \\ \vspace{2mm}
        \frac{B}{2}  & C            & \frac{E}{2}  \\
        \frac{D}{2}  & \frac{E}{2}  & F \\		
    \end{bmatrix}
\]%

Hence the conic is produced by:

\[%
    \begin{bmatrix}
        x & y & 1 \\
    \end{bmatrix}
    \begin{bmatrix}\vspace{2mm}
        A            & \frac{B}{2}  & \frac{D}{2}  \\ \vspace{2mm}
        \frac{B}{2}  & C            & \frac{E}{2}  \\
        \frac{D}{2}  & \frac{E}{2}  & F \\		
    \end{bmatrix}
    \begin{bmatrix}
        x \\		y \\
        1 \\
    \end{bmatrix}
    = 0
\]%

Which expands to:

\[%
    Ax^{2}+Bxy+Cy^{2}+Dx+Ey+F = 0
\]%

\end{document}

