\documentclass{article}
\author{Alex Hiller}
\title{}
% Type-setting
\setlength{\parindent}{0cm}
\setlength{\parskip}{0.125cm}
\pagenumbering{gobble}
\usepackage[margin=0.5cm]{geometry} % Formatting
\usepackage{amsmath}      % Mathematics
\usepackage{amssymb}      % Mathematics
\usepackage{listings}     % Listings
\usepackage{esint}        % Mathematics
\usepackage{color}        % Listings
\usepackage{courier}      % Listings
\usepackage{circuitikz}   % Circuits
\usepackage{titlesec}     % Section Formatting
\usepackage{stmaryrd}               % \mapsfrom arrow. 
\input{/home/polluticorn/GitHub/texTemplates/texMacros}
% Section formatting
\titleformat{\section}{\huge \bfseries}{}{0em}{}[]
\titleformat{\subsection}{\Large \bfseries}{}{0em}{}
\titleformat{\subsubsection}{\bfseries}{}{0em}{}

%%%%%%%%%%%%%%%%%%%%%%%%%%%%%%%%%%%%%%%%%%%%%%%%%%%%%%%%%%%%%%%%%%%%%%%%%%%%%%%%

\begin{document}

\section{Linear Independence} 
\subsection{Theorem} 
Any set $\{ v_1,v_2,\ldots,v_{p}  \} \in \R^{n} $ is linearly independent if $p>n$.

\section{Subspace}
\[
    H \text{ is a subspace of } V
\]%
Then:

$H$ contains the zero vector of $V$

$\ldots$  

\section{Subspaces spanned by a Set} 
\[
    \{ v_{1} , v_{2} , \ldots v_{n} \}  \in V \Rightarrow \text{Span} \{ v_{1} , v_{2} , \ldots v_{n} \}
    \text{ is a subspace of } V
\]%


\section{Linearly Independent Sets} 
There is no scalar such that:
\[
    \cos( t ) = c \sin( t ) 
\]%
This leads to:
\[
    \frac{1}{\tan (t) } = c
\]%
You can solve for a particular $t$  value, but not a scalar.

Because of this we can conclude that $ \cos( t )  $  is not a scalar multiple of
$ \sin( t)  $ and hence they are not linearly related.


\section{Basis} 
Let $ H $  be a subspace of $ V $. \\
You have an indexed set $ \mathbf{B} $.
\[
    \mathbf{B} = \{ b_1,b_2,\ldots,b_{n}  \} 
\]%
If 
\[
    \mathbf{B} \in V
\]%
And $ \mathbf{B} $  is linearly independent, and:
\[
    H = \text{Span} \{ b_1,b_2,\ldots,b_{n}  \} 
\]%
Then $ \mathbf{B} $  is a basis in $ H $.


Note that if $ H = V $  this still applies, as all vector spaces are a subspace
of themselves.

\clearpage

\section{Basis Examples} 
A vector space can have multiple basis vectors, though they all have the same
number of components -- this number is called the dimension.

If you remove one vector from the basis, the space will not be fully spanned, if
you add an extra vector in then you lose linear independence.

\section{Why Impose a Basis} 
You may need a coordinate system for your vector space.
\qanda{What are the axioms of a vector space?}{}

\section{Proof of the Unique Representation Theorem} 
Is done by a proof-by-contradiction.

You make an opposite assumption and arrive at a contradiction $ \Rightarrow  $
opposite must be true.

\qanda{Does a proof by contradiction require that there only be two options to
assume from? Or does a proof by contradiction just lead to an exclusion of the
assumed case?}{} 



\end{document}
