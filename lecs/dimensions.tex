\documentclass{article}
\author{Alex Hiller}
\title{}
% Type-setting
\setlength{\parindent}{0cm}
\setlength{\parskip}{0.125cm}
\pagenumbering{gobble}
\usepackage[margin=2.5cm]{geometry} % Formatting
\usepackage{amsmath}      % Mathematics
\usepackage{amssymb}      % Mathematics
\usepackage{listings}     % Listings
\usepackage{esint}        % Mathematics
\usepackage{color}        % Listings
\usepackage{courier}      % Listings
\usepackage{circuitikz}   % Circuits
\usepackage{titlesec}     % Section Formatting
\usepackage{stmaryrd}     % \mapsfrom arrow. 
\usepackage{mathtools}    % \coloneqq
\usepackage{svg}
\setsvg{inkscape=inkscape -z -D}
\input{/home/polluticorn/GitHub/texTemplates/texMacros}
% Section formatting
\titleformat{\section}{\huge \bfseries}{}{0em}{}[]
\titleformat{\subsection}{\Large \bfseries}{}{0em}{}
\titleformat{\subsubsection}{\bfseries}{}{0em}{}

%%%%%%%%%%%%%%%%%%%%%%%%%%%%%%%%%%%%%%%%%%%%%%%%%%%%%%%%%%%%%%%%%%%%%%%%%%%%%%%%

\begin{document}

\section{Dimensions} 
\subsection{Standard Unit Vectors} 
Standard unit vectors for $ \R^{3} $ :
\[%
    \mathbf{e}_{1} 
    =
    \begin{bmatrix}
        1 \\ 0 \\ 0 \\
    \end{bmatrix}
    \quad
    \mathbf{e}_{2} 
    =
    \begin{bmatrix}
        0 \\ 1 \\ 0 \\
    \end{bmatrix}
    \quad
    \mathbf{e}_{3} 
    =
    \begin{bmatrix}
        0 \\ 0 \\ 1 \\
    \end{bmatrix}
\]%
The pattern continues for $ n $-dimensional space.

\subsection{Standard Matrix of $ \mathcal{T} $} 
$ \mathbf{x} \in \R^{2} $ 

Because of linearity:
\[%
    \mathcal{T}(\mathbf{x}) 
    =
    \mathcal{T}(x_1 \mathbf{e}_{1} )
    + 
    \mathcal{T}(x_2 \mathbf{e}_{2} )
    = 
    x_1 \mathcal{T}(\mathbf{e}_{1} ) + x_2 \mathcal{T}(\mathbf{e}_{2} )
\]%
Then we define:
\[%
    \mathbf{x} 
    =
    \begin{bmatrix}
        x_1 \\
		x_2 \\		
    \end{bmatrix}
\]%
and
\[%
    \mathbf{T}
    =
    \begin{bmatrix} \
        \mathcal{T}(\mathbf{e}_{1} ) \ | \
        \mathcal{T}(\mathbf{e}_{2} ) \
    \end{bmatrix}
\]%
\subsubsection{Key point:} 
This allows us to take a vector mapping function ($ \mathcal{T}(\mathbf{x}) $)
and turn it into a matrix equation:
\[%
    \mathcal{T}(\mathbf{x})
    =
    \mathbf{T} \mathbf{x}
\]%

\end{document}

