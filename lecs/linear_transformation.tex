\documentclass{article}
\author{Alex Hiller}
\title{}
% Type-setting
\setlength{\parindent}{0cm}
\setlength{\parskip}{0.125cm}
\pagenumbering{gobble}
\usepackage[margin=2.5cm]{geometry} % Formatting
\usepackage{amsmath}      % Mathematics
\usepackage{amssymb}      % Mathematics
\usepackage{listings}     % Listings
\usepackage{esint}        % Mathematics
\usepackage{color}        % Listings
\usepackage{courier}      % Listings
\usepackage{circuitikz}   % Circuits
\usepackage{titlesec}     % Section Formatting
\usepackage{stmaryrd}               % \mapsfrom arrow. 
\usepackage{svg}
\setsvg{inkscape=inkscape -z -D}
\input{/home/polluticorn/GitHub/texTemplates/texMacros}
% Section formatting
\titleformat{\section}{\huge \bfseries}{}{0em}{}[]
\titleformat{\subsection}{\Large \bfseries}{}{0em}{}
\titleformat{\subsubsection}{\bfseries}{}{0em}{}

%%%%%%%%%%%%%%%%%%%%%%%%%%%%%%%%%%%%%%%%%%%%%%%%%%%%%%%%%%%%%%%%%%%%%%%%%%%%%%%%

\begin{document}

\maketitle
\section{How Matrix Multiplication is a Mapping} 
A matrix multiplication can be considered as a linear transformation or an
operation on a vector, in the form:
\[%
    \mathbf{x_{new} } = \mathbf{A}\mathbf{x}
\]%
The input vector, $ \mathbf{x} $ is operated on by the matrix $ \mathbf{A} $.

Assume that $ \mathbf{A} $ only operates on $ m \times 1 $ vectors, then:
\[%
    \mathbf{A} \in \R^{n \times m}
\]%
\[%
    \mathbf{A}: \R^{m} \rightarrow \R^{n}
\]%
This is because the input vectors will have $ m \times 1 $ dimension, meaning
the $ \mathbf{A} $ operation maps it over to $ n \times 1 $ space, just by the
dimensions of $ \mathbf{A} $ and elementary linear algebra.

\section{Why $ \mathbf{A}\mathbf{x}+\mathbf{q} $ is not a linear
transformation}
For a tranfsformation to be linear, it must obey:
\[%
    \mathcal{T}(c \mathbf{v}) = c \cdot \mathcal{T}(\mathbf{v})
\]%
If we take $ c=0 $, then:
\[%
    \mathbf{A}\left(0 \mathbf{x}\right)+ \mathbf{q} \neq 0
    \left(\mathbf{A}\mathbf{x}+\mathbf{q}\right)
\]%
\[%
    \mathbf{q} \neq 0
\]%
Of course if \textbf{q} is not 0, then it is a linear transformation, which we
looked at as a mapping, above.

\section{Projection Transformation} 
You can project one vector space onto another one:
\[%
    \begin{bmatrix}
        x_1 \\
		x_2 \\
		x_3 \\		
    \end{bmatrix}
    \mapsto 
    \begin{bmatrix}
        1 & 0 & 0 \\
		0 & 1 & 0 \\
		0 & 0 & 0 \\		
    \end{bmatrix}
    \begin{bmatrix}
        x_1 \\
		x_2 \\
		x_3 \\		
    \end{bmatrix}
    = 
    \begin{bmatrix}
        x_1 \\
		x_2 \\
		0 \\		
    \end{bmatrix}
\]%
Hence it projects $ \R^{3}\rightarrow \R^{2} $ by simply destroying the
third-dimension coordinate.

%
\qanda
{Would this be the way that higher dimensions are visualised? By just
eliminating their higher coordinates ($ x_4,x_5,\ \ldots \, x_{n} $)? }
{j} 

%
\clearpage
\section{Theorem} 
A linear transformation $ \mathcal{T}: \R^{n} \mapsto \R^{m}  $ is a one-to-one
transformation iff the equation $ \mathcal{T}(\mathbf{x})= \mathbf{0} $ has only
the trivial solution: $ \mathbf{x}=\mathbf{0} $.

\section{Theorem} 
Let $ \mathcal{T}: \R^{n} \mapsto \R^{m} $ be a linear transformation with the
standard matrix $ \mathbf{T} $. 

\subsubsection{Result 1:}

$ \mathcal{T} $ is a map to $ \R^{m} $ iff the columns of $ \mathbf{T} $ 
span $ \R^{m} $. 

\subsubsection{Result 2:}

$ \mathcal{T} $ is one-to-one iff the columns of $ \mathbf{T} $ are linearly
independent. 

\subsubsection{Comments:} 

Result 1 $ \perp $ Result 2 -- i.e. can have one without the other.

E.g. You can have a one-to-one mapping without spanning the codomain.

You can also span the codomain and not have a one-to-one mapping by having $
\mathbf{T} : \R^{3}\rightarrow \R^{2} $.





\end{document}
