\documentclass{article}
\author{Alex Hiller}
\title{}
% Type-setting
\setlength{\parindent}{0cm}
\setlength{\parskip}{0.125cm}
\pagenumbering{gobble}
\usepackage[margin=2.5cm]{geometry} % Formatting
\usepackage{amsmath}      % Mathematics
\usepackage{amssymb}      % Mathematics
\usepackage{listings}     % Listings
\usepackage{esint}        % Mathematics
\usepackage{color}        % Listings
\usepackage{courier}      % Listings
\usepackage{circuitikz}   % Circuits
\usepackage{titlesec}     % Section Formatting
\usepackage{stmaryrd}     % \mapsfrom arrow. 
\usepackage{mathtools}    % \coloneqq
\usepackage{svg}
\setsvg{inkscape=inkscape -z -D}
\input{/home/polluticorn/GitHub/texTemplates/texMacros}
% Section formatting
\titleformat{\section}{\huge \bfseries}{}{0em}{}[]
\titleformat{\subsection}{\Large \bfseries}{}{0em}{}
\titleformat{\subsubsection}{\bfseries}{}{0em}{}

%%%%%%%%%%%%%%%%%%%%%%%%%%%%%%%%%%%%%%%%%%%%%%%%%%%%%%%%%%%%%%%%%%%%%%%%%%%%%%%%

\begin{document}
\section{Basis} 
Must be linearly independent set.

\section{Spanning Set Theorem} 
If you have a set $ \mathcal{S} $ and it contains some vectors $ \left\{
\mathbf{v_1}, \mathbf{v_2} , \ \ldots \ \mathbf{v}_{n}  \right\} $
And you know that $ \mathcal{S}  $ spans $ \mathcal{H}  $, then if you remove
the dependent vector from $ \mathcal{S}  $ then it will still span $ \mathcal{H}  $ 

\section{Dimension} 
\[%
    \text{dim}(V) = n
\]%
It means the number of vectors in the basis of $ V $.

\subsection{Note:} 
\[%
    \text{dim}(\mathbf{0}) = 0
\]%
If $ V $ is not spanned by a finite set then it is $ \infty  $ dimensional.

\section{Basis Theorem} 
$ V $ is a $ p $ dimensional vector space. 

$ p \ge 1 $ 

Then:

Any combination of $ p $ linearly independent elements in V can form a basis in
$ V $.

Any set of $ p $ elements that spans $ V $ is a basis for $ V $.

\subsection{Takeaway:} 
There are many ways to form a basis.









\clearpage
-
\end{document}

